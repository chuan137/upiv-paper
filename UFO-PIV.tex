\documentclass[11pt,draft]{amsart}
\usepackage{geometry}           % See geometry.pdf to learn the layout options. There are lots.
\geometry{a4paper}              % ... letterpaper or a4paper or a5paper or ... 
%\geometry{landscape}           % Activate for for rotated page geometry
\usepackage[parfill]{parskip}   % Activate to begin paragraphs with an empty line rather than an indent
\usepackage{graphicx}
\usepackage{amssymb}
\usepackage{epstopdf}
\DeclareGraphicsRule{.tif}{png}{.png}{`convert #1 `dirname #1`/`basename #1 .tif`.png}

\title{Towards an Online Analysis for Defocusing Micro Particle Tracking Velocimetry}
\author{C. Miao and A. Lewkowicz and M. Vogelgesang and T. Dritschler and S. Chilingaryan and A. Kopmann and P. Cavadini and H. Weinhold}
\address{Karlsruhe Institute of Technology, Postfach 3640, D-76021 Karlsruhe, Germany}
\date{2.7.2015}                     % Activate to display a given date or no date


\begin{document}
\maketitle

\begin{abstract}

At KIT a new micro-PIV setup is currently developed. 
It is intended to analyze flow field in thin films. 

Defocusing particle tracing is able to detect the position of traced particles in a certain volume around the focal plane. The setup exploits this technique and used unto five
planes in parallel. 

In order to evaluate the experiments and to enable feedback control online reconstruction is currently developed. Lastest advances in parallel computing might give rise accelleration factors up to the real-time domain. 

The article will present hardware and software environment and present a detailed analysis of off-focus PIV algorithms. Latest benchmark results of the project are presented.

\smallskip
\noindent \textbf{Keywords.} Online Monitoring, GPU Computing, Defocusing PIV
\end{abstract}



\section{Introduction}

Who proposed the defocusing method the first time? Sample pictures of the method. 
Speidel et al. (2003), Wu et al. (2005), Park und Kihm (2006)


Overview of the different communities and application that use the method


Previous work of of other groups, available open source applications, commercial applications?

Why does optimization of standard applications make sense?


Measures to accelerate analysis of PIV code. Estimate individual optimization potential. 
\begin{itemize}
\item M1: Optimized algorithm 10-50
\item M2: Optimized implementation for parallel execution 10-20
\item M3: Usage of GPU as co-processors (instead or in addition to CPU) 5-10
\item M4: Optimization for certain GPU architecture 2-4
\item M5: Usage of GPU Cluster 2-4
\item M6: New hardware generation 2-3
\item M7: Management of spatial and temporal resolution (for online monitoring) 4-16
\end{itemize}



\section{Hardware and software environment}

\subsection{Experimental setup}

Short description of the setup. Picture of the observation volume. 

Description of reference datasets (put these datasets also to UFO server to allow comparison of algorithms)


\subsection{The UFO Parallel Computing Framework}

Design principles.

Advantages for the developer

Advantages for the algorithm designer

Interface for the user, options for integration


\section{PIV Algorithm}

Describe analysis chain. Block diagram. Profiling.
Display variants of the algorithm as branches.


\subsection{Image pre-processing}

\subsection{Particle detection}

\subsection{Particle properties}

\subsection{Tracking}



\section{Results}

\subsection{Performance}

Present different versions of the code.

\begin{itemize}
\item (1) 4DTracker: MatLab code, ordered filter approach, single core, no optimization, by Wu et. al
\item (2.1) "FastTracker": ordered filter code, optimized for parallel execution on CPU, by T Habel and A. Lewkowicz
\item (2.2) "FastTracker": ordered filter code, optimized for parallel execution on GPU, by T Habel and A. Lewkowicz
\item (3.1) UFO-PIV: Hough transform, optimized for GPU, executed on CPU, by A. Lewkowicz and C. Miao and the UFO-team
\item (3.2) UFO-PIV: Hough transform, optimized for GPU, executed on GPU (Cluster), by A. Lewkowicz and C. Miao and the UFO-team
\end{itemize}

Comparison of the different codes. Relate to the optimization measures above. 

Measure M1: 2.1 and 3.1 (for comparison do also 2.2 and 3.2) are both optimized. The difference is the better suited algorithm in 3. 

Measure M2: 1 and 2.1 have the same algorithm, but 2.1 used parallel optimization for CPUs. 
Compare with the Ratzenbook 2013 code. They exploit M1 and M2. Run our code 3 on CPU only.
Should we introduce code 3.0 on GPU for this purpose?

Measure M3: Compare 2.1 and 2.2 (or 3.1 and 3.2), same implementation but execution on CPU and GPU. 

Measure M4: Not used currently (my be report experience form PyHST?). (csa)

Measure M5: Execute 3.2 at a GPU cluster. The performance of the execution is limited by the number of GPUs. For the 5 camera setup the demand for GPUs scales by additional factor of 5, that more than 4 GPUs per camera brach seem not feasible. 

Measure M6: Execute 3.1 and 3.2 at different CPUs / GPUs and plot versus the release dates of the GPUs. Add roadmap data from the manufactures. 

Check if some measures can be plotted in the same graph? 


\subsection{Precision}

Evaluation of detection efficiency, concerning false positive and false negatives. Compare with 
Ratzenbook 2013.





\subsection{Particle management}

Measure M7: Analyse quality of particle detection with quality of the resulting trajectory.
Analyst how many particles are needed in a volume and time slice in relation to the velocity to get a good quality of the flow field.



\section{Conclusion}

The UFO platform is ideal to create workflows with optimized algorithms. 

UFO-PIV code is the fastest code and available online. It can be easily customized for other setups. 

Public high-quality code is crucial of the development of cutting edge experiments in all communities. Only with a common effort maintainable and reliable code can be ensured in the future. 

Online monitoring seems to the feasible, while not completely reached. The group will proceed to develop the first online PIV setup with 3D visualization in a volume of about 100um x 100 um x 100 um.

Future plans, next steps to develop the aimed online monitoring.
 
Scientific GPU computing is affordable and can replace tedious computations for many applications. Beside PIV there are other applications like tomography where GPUs have been successfully used.

Link the the UFO-PIV code repository or website. Publish also the sample dataset used in the article. The UFO-PIV code and the dataset should be stored at GitHUB / zenodo.org with an DOI.



\section{Acknowledgement}

The group of Wu for the initial code 4DTracker.

The UFO framework has been initiated due to UFO Grants and has been developed to universal framework for online monitoring systems for high data-rate since then.

\end{document}  

