\documentclass[11pt]{amsart}
\usepackage{geometry}           % See geometry.pdf to learn the layout options. There are lots.
\geometry{a4paper}              % ... letterpaper or a4paper or a5paper or ... 
%\geometry{landscape}           % Activate for for rotated page geometry
\usepackage[parfill]{parskip}   % Activate to begin paragraphs with an empty line rather than an indent
\usepackage{graphicx}
\usepackage{amssymb}
\usepackage{epstopdf}
\DeclareGraphicsRule{.tif}{png}{.png}{%
    \noexpand\epstopdfcall{convert #1 \noexpand\OutputFile}%
}

\usepackage{cite}
\usepackage{hyperref}
\usepackage[modulo]{lineno}
\linenumbers

\title{Towards an Online Analysis for Defocusing Micro Particle Tracking Velocimetry}
\author{C. Miao and A. Lewkowicz and M. Vogelgesang and T. Dritschler and S. Chilingaryan and A. Kopmann and P. Cavadini and H. Weinhold}
\address{Karlsruhe Institute of Technology, Postfach 3640, D-76021 Karlsruhe, Germany}
\date{2.7.2015}                     % Activate to display a given date or no date

\begin{document}
\maketitle

\begin{abstract}

At KIT a new micro-PIV setup is currently developed.  It is intended to analyze
flow field in thin films. 

Defocusing particle tracing is able to detect the position of traced particles
in a certain volume around the focal plane. The setup exploits this technique
and used unto five planes in parallel. 

In order to evaluate the experiments and to enable feedback control online
reconstruction is currently developed. Lastest advances in parallel computing
might give rise accelleration factors up to the real-time domain. 

The article will present hardware and software environment and present a
detailed analysis of off-focus PIV algorithms. Latest benchmark results of the
project are presented.

\smallskip
\noindent \textbf{Keywords.} Online Monitoring, GPU Computing, Defocusing PIV, 
Hough Transform, Statistical
\end{abstract}


Who proposed the defocusing method the first time? Sample pictures of the method. 
Speidel et al. (2003), Wu et al. (2005), Park und Kihm (2006)


Overview of the different communities and application that use the method


Previous work of of other groups, available open source applications, commercial applications?

Why does optimization of standard applications make sense?


Measures to accelerate analysis of PIV code. Estimate individual optimization potential. 
\begin{itemize}
\item M1: Optimized algorithm 10-50
\item M2: Optimized implementation for parallel execution 10-20
\item M3: Usage of GPU as co-processors (instead or in addition to CPU) 5-10
\item M4: Optimization for certain GPU architecture 2-4
\item M5: Usage of GPU Cluster 2-4
\item M6: New hardware generation 2-3
\item M7: Management of spatial and temporal resolution (for online monitoring) 4-16
\end{itemize}




\section{Methodology} 

For ring detection, we use the circular Hough Transform (CHT).  Since the
defocued ... method, ring edges are thick, fuzzy and have rather low contrast
against the background.  we have skipped the edge detection, apply HT directly
to the input image, following contrast adjustment.  As a result, the HT votes
are contaminated by pixels that does not belong to any shape and are therefore
very noisy. For the purpose of efficiently and reliably sepearte the signal
from noises, we used a likelihood ratio function to filter the votes.  And
later for determining the ring sizes, radical histogram algorithm is used.

\subsection{Pre-processing}
we apply a contrast adjustment, remove background noise would help.
from histogram background noise gaussian distribution, signal in tail
we find the position and width of the histogram peak, and remove all pixels below threadshold
$$s = I_{\mathrm{peak}} + n *\sigma_\mathrm{peak}$$
n is a parameter, which is usally set to 1.

\subsection{Hough Transform: ring templates} 

Circular Hough transform, maps the input image into accumulator space, x, y and
r.  Although the space is continous, in practice it is divided into discrete
cells.  Each cell corrsponds to superpositions of all ring shapes from postions
included in the cell.  We can view the CHT as a approach combining template
matching and HT. We use ring template with various ring size, covolve with
input image. The covolution is used as votes in accumulator space.

We have studied the impacts of ring templates on the voting. For a step fucntion,
..., side slobes. Therefore we use a skewed function..., ..... To account for the
fuzzy egeds, we gaussain shape of the skewed function,...

\subsection{Likelihood ratio}
Hough transform.
common strategy is to apply an edge detection before HT.
Images of micro particles are fuzzy, does not have a crisp edges. 
Signals can be very close to the background.
We imporve the contrast to gain higher SNR instead of edge detection.

\begin{figure}
\centering
\includegraphics[width=0.35\textwidth]{figs/image1.png}\qquad
\includegraphics[width=0.35\textwidth]{figs/img0-crop.png}
\label{fig:input}
\caption{Picture captured in experiment.}
\end{figure}


contrast, histogram, singmoid, background fitting

\begin{figure}
\centering
\includegraphics[width=0.25\textwidth]{./figs/img0-crop-contrast.png}
\includegraphics[width=0.35\textwidth]{figs/img0-crop.png}
\end{figure}


Likelyhood function. 
\begin{equation}
    L(\pi_{ij}) = \exp\left( -\frac{1}{\sigma_n^2 } \sum \right)
\end{equation}




\end{document}
